\section{Benchmarks}

\subsection{Methodology}

Each problem is timed with all combinations of optimization flags.\\
Timeout is set to 30 seconds, any execution error or timeout results in an
\(\infty\) for this combination.\\

Average is calculated on 5, 20 or 50 runs depending on execution time, ensuring
that the displayed average has sufficient precision.\\

To avoid interference, all benchmarks are run consecutively on an otherwise
inactive computer.\\


\subsection{Results}

\begin{center}\label{bench-results}
\input{bench-results}
\end{center}

The main takeaway is that none of the optimizations are useless :
\ttt{1} yields a small but noticeable improvement in performance on
\ttt{flake5h}, \ttt{scale4}, \ttt{scale5}, \ttt{large1}, \ttt{large2}, \ttt{trefoil};
\ttt{X} is useful in the same way for \ttt{scale2}, \ttt{scale3}, \ttt{flake4hv}.\\
Finally, \ttt{T} provides quite spectacular improvements for \ttt{branches},
\ttt{large4}, \ttt{trefoil}, \ttt{trimtriple} and slightly less so for \ttt{large3}.
Only in the pathological case \ttt{display} does \ttt{T} hinder performance.\\

Overall, these benchmarks suggest choosing \ttt{-o1XT} for best
(and most consistent) results, which happens to be the default when \ttt{-o} is not
specified.\\

