\section{Usage}\label{sec-usage}

\subsection{\ttt{make} targets}

By default, \ttt{make} builds the \ttt{pingouin} executable.\\
\ttt{make test} will run unit tests for all functions of
\ttt{Priority} and \ttt{Bitset} that are part of the original
interface (see the documentation for more details).\\

\ttt{make doc} and \ttt{make report} build the documentation and the
\ttt{README.pdf} respectively.\\

Some other tools are
\begin{itemize}
    \item \ttt{tester.py} which performs path length checks on all files in \ttt{problems/}
    \item \ttt{bencher.py} to produce benchmarks
    \item \ttt{reporter.py} to translate benchmark results to a formatted tabular
\end{itemize}

Note that source code has been moved to \ttt{src/}, the \ttt{Makefile} has
been adapted in consequence.

\subsection{Command-line arguments}

\textit{All of the following is accessible in more detail by running
\ttt{./pingouin -h}}\\

\ttt{pingouin} will accept at most one positional argument being the name of the
file that contains the problem. If this argument is not provided the problem will
be read from \ttt{stdin}.\\

Named argument take the form \ttt{-[CATEGORY][FLAGS]}\\
where \ttt{CATEGORY} is one of \ttt{o} (optimizations), \ttt{d} (display),
\ttt{h} (help).\\

Example (see the optimizations section for details):\\
There are three optimizations available: \ttt{1}, \ttt{X}, \ttt{T}.\\
Running \ttt{./pingouin -o1X file} will enable optimizations \ttt{1} and \ttt{X}
but disable \ttt{T}.\\
\ttt{d} will use its default configuration, as would have \ttt{o} were it not
explicitly specified.\\

\ttt{h} is special in that its presence will cause other arguments to be ignored
and the program will terminate after printing the help message.\\


