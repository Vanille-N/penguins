\section{Optimizations}


\subsection{\ttt{1}, \ttt{X}}

In many of the problems an optimal path, or at least a path close to the optimum,
can be obtained by only using ``single'' moves (i.e. of the form \ttt{(d,1)}),
or ``extremal'' moves (i.e. of the form either \ttt{(d,1)} or \ttt{(d,n)} when
\ttt{(d,n+1)} is not allowed).\\

It has been observed (see benchmarks section) that searching for an optimal
solution with such a restriction of moves often improves performance.\\
The added cost of exploring a whole tree of configurations is compensated
by the fact that this tree smaller than the full tree of allowed moves
enables removing many branches that are such terrible configurations
that it is possible to do better even with more restrictions on allowed moves.\\

This is implemented in two variants that can be enabled with the option flags
\ttt{-o1} (``single'' moves) and \ttt{-oX} (``extremal'' moves), or even
\ttt{-o1X} (both, with ``single'' first and ``extremal'' next).\\


\subsection{\ttt{T}}

The above optimization (and the algorithm as a whole) is particularly effective
when the optimal path is perfect, i.e. goes through all positions, because
as soon as a perfect path is found the whole exploration can be stopped.\\
This leads to the purpose of this optimization : to trim positions that
are useless to reach in order to make it easier to eliminate branches of the
exploration tree.\\

Procedure:
\begin{itemize}
    \item find all positions that create a one-way split, i.e. have a position
    that satisfies the two following conditions: (a) it splits the board into
    two or more zones, and (b) it is impossible to go from a zone to another
    without stepping on the position.
    \item for each of these positions, select the zone that does not
    contain the starting point
    \item order these zones by inclusion
    \item calculate the best possible path for each zone (memoize the results
    for subsets) and delete all positions that are not part of the optimal path
\end{itemize}~\\

\ttt{large3} is a good example of how this optimization operates.\\
Initial configuration:
\begin{lstlisting}
     * * *   * *       
    * # * * *         
   * * * *   *         
    * * * *   *       
     * * * *   * *     
\end{lstlisting}

There are three positions that create a split as described, marked here:
\begin{lstlisting}
     . . .   * .       
    . . . . *         
   . . . .   .         
    . . . .   .       
     . . . .   * .
\end{lstlisting}

Two of them have optimal paths that are trivial to calculate:
\begin{lstlisting}
     . . .   a b       
    . . . . .         
   . . . .   .         
    . . . .   .       
     . . . .   . .
\end{lstlisting}
and
\begin{lstlisting}
     . . .   . .       
    . . . . .         
   . . . .   .         
    . . . .   .       
     . . . .   a b
\end{lstlisting}

The third uses the memoized calculations from the other two to quickly compute
the best path
\begin{lstlisting}
     . . .   * *       
    . . . . a         
   . . . .   b         
    . . . .   c       
     . . . .   d e
\end{lstlisting}

The two unused positions are deemed useless, and once they are removed
the main algorithm will have no difficulty quickly computing a path
for the remaining positions, once again reusing the memoized results.
\begin{lstlisting}
     e d c   . .
    f a b s t
   h g q r   u
    i j p o   v
     k l m n   w x
\end{lstlisting}
This last computation is efficient is major part because the optimal path
is perfect once the useless positions are deleted.


